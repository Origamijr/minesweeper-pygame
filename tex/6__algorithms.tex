\chapter{Algorithms}\label{sec:algorithms}

There are limits to what humans are able to compute, but for computers, those limits are much higher. Although this document aims to serve as a guide for human play, computer simulations able to compute probabilities described in Section \ref{sec:guessing} may aid us by adding guessing strategies to our repertoire. This section will cover the data structures and algorithms that aid in probability calculation and logical play. These form the building blocks for what one would need to build a minesweeper solver.\\

Now one may ask, ``Why are we discussing computer algorithms in a document for human play?" Well, the answer is three-fold. First, a computer can obviously play minesweeper much better than a human. Although, at the moment, a computer cannot play perfectly, they can closely approximate perfect play. This provides a good benchmark to compare human strategies against. Second, evaluating human play is very difficult without a lot of data. With a computer however, we can simulate human strategies many times to evaluate how good a human strategy is compared to the computer benchmark (although if a large amount of human play data could be obtained, I think some more interesting analysis can be performed). Finally, I have minesweeper code and I wanted to use it, so I'm subjecting my reader to the coding process I used for my simulations.\\

\subsubsection*{Code where?}
Code here\footnote{https://github.com/Origamijr/minesweeper-pygame}. The repository does not contain code for all the algorithms discussed in this chapter, just the ones I felt like implementing. For obvious reasons, python is not the ideal choice for implementing these algorithms due to algorithm runtimes. However, as I am most familiar with python, I believe that it would be the language that I would be able to keep the code as short as possible while still being somewhat human readable.\\

\section{Single Mine Probability}
The first algorithm of interest to most programmers who play minesweeper is usually to determine the probability of a square containing a mine. We'll call this function we want to implement \texttt{Probability}, which takes a board and returns a grid of the same size filled with corresponding probabilities for each square. As we saw in Section \ref{sec:probability}, this would normally involve enumerating the natural complete boards that complete a current board. We will see how that can be done, and if there's any optimizations we can make to make this problem for tractable.\\

\subsection{Exhaustive Approach}

The easiest to understand approach to this problem is to compute mine probability directly from the definition itself, by enumerating all of the solutions.\\

Although there are many ways to enumerate the solutions, I want to call to attention one specific method to possibly enlighten readers on new mathematical insight.\\

One particular interesting formulation of minesweeper was made by Fowler and Young\footnote{https://minesweepergame.com/math/minesweeeper-a-statistical-and-computational-analysis-2004.pdf} in 2004 using linear algebra. Recall that an MSM $A$ is a set of square for which we know the number of mines contained within the set. In Section \ref{sec:theorems}, we saw that $M(A)\mapsto k$ this yielded the equation $\sum_{a\in A}M'(a)=k$ for all valid complete boards that continue our current board.\\

The trick is to notice that these equations are linear with $|\mathcal{A}|$ variables given by $m_a=M'(a)$ for each $a\in\mathcal{A}$. In order to enforce minecount, we just need one more equation $\sum_{a\in\mathcal{A}}M'(a)=n$. All we need need to do is create an equation for each first order MSM from Lemmas \ref{thm:trivial_msm} and \ref{thm:number_msm}. From this, we have a set of linear equations, for which any solution would correspond to a valid board completion.\\

TODO. example\\

However, as one would expect, there are a lot of solutions to this system of equations. We only start with as many equations as we have numbers and flags plus one for the minecount, and any MSM discovered through second order equation does not yield new information since they come from known equations. This often leads to an undercomplete system since we have $|\mathcal{A}|$ unknowns, and often not that many equations when it matters.\\ 

If one had the willpower (and the compute power) to iterate through all $2^{|\mathcal{A}|}$ possible solutions, then computing the single mine probability is easy. Just count the number of solutions with a mine in that square, and divide it by the total number of mines. In pseudocode, this approach is given by the following pseudocode.

\begin{algorithm}[h]
\caption*{Naive Single Mine Probability through Exhaustive Search}
\begin{algorithmic}
\Function{\texttt{Probability}}{$B=(\mathcal{A},n,M,C,N)$}
\State{$A\gets[[1,1,\dots,1]]$}\Comment{First row is $\sum_{a\in\mathcal{A}}M'(a)=n$}
\State{$b\gets [n]$}\Comment{First row is $\sum_{a\in\mathcal{A}}M'(a)=n$}
\For{$a\in\mathcal{A}$}
    \If{$M(a)=1$}\Comment{Append trivial MSMs $M(a)\mapsto1$ if $M(a)=1$}
        \State{$A$.append\_row([``1 at column corresponding to $a$, 0s elsewhere"])}
        \State{$b$.append(1)}
    \EndIf
    \If{$C(a)=1$}\Comment{Append trivial MSMs $M(a)\mapsto0$ if $C(a)=1$}
        \State{$A$.append\_row([``1 at column corresponding to $a$, 0s elsewhere"])}
        \State{$b$.append(0)}
    \EndIf
    \If{$N(a)\in\mathbb{Z}$}\Comment{Append number induced MSMs $M(K(a))\mapsto N(a)$}
        \State{$A$.append\_row([``1s corresponding to neighbors of $a$, 0s elsewhere"]}
        \State{$b$.append($N(a)$)}
    \EndIf
\EndFor
\State{Solutions $\gets\emptyset$}
\For{$x\in\mathbb{R}^{|\mathbb{A}|}$}
    \If{$Ax=b$}
        \State{Solutions.append($x$)}
    \EndIf
\EndFor
\For{$a\in\mathcal{A}$}
    \State{$P(a)\gets$ (\# of solutions with $a$)/solutions.size}
\EndFor
\Return $P$
\EndFunction
\end{algorithmic}
\end{algorithm}

Finding all the solutions that satisfy the linear system takes exponential time, but can be optimized in various ways.First, one can notice that the minecounting equation actually restricts us to $|\mathcal{A}|\choose n$ solutions, which is still exponential, but smaller than $2^{|\mathcal{A}|}$. We can also notice that we can use Gaussian elimination to reduce the matrix to a smaller block matrix, and discover free variables which can either contain a mine or be clear. In effect, solution enumeration would require discovering what the free variables are, and pruning solutions in the solution tree from which a contradiction arises. This procedure has a lot of intricacies, so I will leave the details for how solutions can be pruned in this approach to the recent paper by Liu et al\footnote{\url{https://minesweepergame.com/math/a-solver-of-single-agent-stochastic-puzzle.pdf}}.\\

I don't want to mislead the reader into believing that the linear algebra approach is inherently slow. In fact, the compact representation of matrices and vectors probably makes this approach the fastest in practice by applying some of the optimizations discussed in the next few sections. However, this compactness leads to a slightly more challenging readability and interpretability for each algorithm. As such, instead I will discuss another common solution enumeration approach which is more ``minesweeper player" interpretable.\\

\subsection{Perimeter Enumeration Approach}

Of course, if the number of unknown squares is large, we can do much better than considering all the solutions. Instead of considering the solutions over the entire unknown region, consider solutions just along the disjoint perimeter sets of the unknown region. Since this is the first discussion of the perimeter of a board, I will give the definition.\\

\defn{Board Perimeter}{\index{perimeter}
The \textbf{perimeter} of a board $B\in\mathbb{B}$ is given by the set\begin{align*}
    \mathcal{P}(B)=U-U_C=\bigcup_{\alpha\in\mathcal{A}}K_U(\alpha)
\end{align*}
where $\alpha\in\mathcal{A}$ such that $N(\alpha)\in\mathbb{Z}$.
}

In English, the perimeter is the complement of the complement MSM, or in other words, the set of unknown squares adjacent to a number. In this sense, the perimeter is the set of squares such that we can infer some local constraint on which solutions the square contains a mine.\\

Once we know the perimeter, we can then try to find all solutions over the perimeter. Once we have the solutions, we can group the solutions we have by the number of mines in them. Preimeter solutions that require more mines than $n$ are invalid, and solutions that have less mines than $n-|U_C|$ are invalid.\\

From there we just need to count the number of solutions over the entire unkown space, including the complement MSM $U_C$. Since $U_C$ is not adjacent to any numbers, a solution over $U_C$ can be any combination of its size and the number of mines to be placed in $U_C$. However, if we know $n_\mathcal{P}$ mines are in a perimeter solution, we can determine the number of mines to be placed in the Complement MSM to be $n-n_\mathcal{P}$. For each perimeter solution with $n_\mathcal{P}$, there are then $|U_C|$ choose $n-n_\mathcal{P}$ ways to place solutions. As such, if there are $M_{n_\mathcal{P}}$ perimeter solutions with $n_\mathcal{P}$ mines, then there are $M_{n_\mathcal{P}}$ times $|U_C|$ choose $n-n_\mathcal{P}$ solutions. Considering all solutions, the number of total solutions for a board is then\begin{align*}
    \sum_{n_\mathcal{P}=n-|U_C|}^nM_{n_\mathcal{P}}{|U_C|\choose{n-n_\mathcal{P}}}
\end{align*}
To figure out how many solutions have a mine at $a$, we just need to find the subset of solutions with a mine at $a$ and repeat the above analysis.\\

Although seperating the unknown region into $\mathcal{P}(B)$ and $U_C$, we can still do better. We were able to seperate the perimeter and the complement MSM because they did not share any numbered square in their neighborhood. This means that solutions are independent assuming we don't care about the total number of mines $n$. We can observe that a similar partition can be made over the perimeter itself.\\

If two subsets of the perimeter don't have any common numbered squares in each of their neighborhoods, we can analyze solutions over them seperately, then use a cartesian product when recombing solutions. TODO more detail\\

Now that we have an idea how to count solutions, we still need to know how to obtain all the solutions over an unknown region. For this, simple recursive branch and bound is sufficient. This is illustrated in the subroutine below.

\begin{algorithm}[h]
\caption*{Finding Solutions for a Board Recursively through Branch and Bound}
\begin{algorithmic}
\Function{\texttt{Solutions}}{$B=(\mathcal{A},n,M,C,N)$}
\If{$|U|==0$ or $B$ not valid}
    \Return{[$\emptyset$]}
\EndIf
\State{$a\gets$ random square in $U$}
\State{soln1 $\gets$ Solutions(($\mathcal{A},\bot,M,C\cup\{a\},N$)}
\State{soln2 $\gets$ Solutions(($\mathcal{A},\bot,M\cup\{a\},C,N$)}
\For{soln in soln2}
    \State{soln.append($a$)}
\EndFor
\Return soln1 $\cup$ soln2
\EndFunction
\end{algorithmic}
\end{algorithm}

With this, the procedure to compute single mine probability is shown below.

\begin{algorithm}[h]
\caption*{Single Mine Probability through Perimeter Search}
\begin{algorithmic}
\Function{\texttt{Probability}}{$B=(\mathcal{A},n,M,C,N)$}
\State{$K\gets\emptyset$}
\For{$a\in\mathcal{A}$ such that $N(a)\in\mathbb{Z}$}
    \State{$K$.append($K_U$)}
\EndFor
\State{solutions $\gets\emptyset$}
\State{components $\gets$ Connected components of $K$ edges defined by intersection (BFS/DFS)}
\For{$A$ in components}
    \State{soln $\gets$ \texttt{Solutions}(($\mathcal{A},\bot,M,C\cup A^C,N$))}
    \For{$s_1$ in solutions}
        \State{solutions.remove($s_1$)}
        \For{$s_2$ in soln}
            \State{solutions.add($s_1\cup s_2$)}
        \EndFor
    \EndFor
\EndFor
\State{counts $\gets$ histogram of sizes of solutions}
\State{total, ctotal $\gets0$}
\For{$(n_\mathcal{P}, M_{n_\mathcal{P}})$ in counts}
    \If{$n_\mathcal{P}+|U_C|<n<n_\mathcal{P}$}\State{continue}\EndIf
    \State{total $\gets$ total $+M_{n_\mathcal{P}}\cdot {{|U_C|}\choose{n-n_\mathcal{P}}}$}
    \State{ctotal $\gets$ ctotal $+M_{n_\mathcal{P}}\cdot {{|U_C|-1}\choose{n-n_\mathcal{P}-1}}$}
\EndFor
\For{$a\in\mathcal{A}$}
    \If{$a\not\in U_C$}
        \State{counts $\gets$ histogram of sizes of solutions with mine at $a$}
        \State{$c\gets0$}
        \For{$(n_\mathcal{P}, M_{n_\mathcal{P}})$ in counts}
            \If{$n_\mathcal{P}+|U_C|<n<n_\mathcal{P}$}\State{continue}\EndIf
            \State{$c\gets c+M_{n_\mathcal{P}}\cdot {{|U_C|}\choose{n-n_\mathcal{P}}}$}
        \EndFor
        \State{$P(a)\gets c/$total}
    \Else
        \State{$P(a)\gets$ ctotal/total}
    \EndIf
\EndFor
\Return $P$
\EndFunction
\end{algorithmic}
\end{algorithm}


\subsection{EP Graph Approach}

While the runtime is still exponential within each partition of the perimeter in the last approach, we have to wonder if we can still do better. In fact, we can. The search space for solutions can be reduced even more if instead of searching over squares, we seqrch for solutions over equiprobable regions.\\

I'll be honest, this idea came from talking with MSCoach, and I haven't implemented it myself.

\defn{Equiprobability (EP) Set}{
\textbf{Equiprobability Sets} $G_1,\dots,G_n$ are a partition of $U$ such that $\forall k$, all $n$-th order mine probabilities are invariant within $G_k$. In other words, $\forall a,b\in G_k$ and $\forall c_1,\dots,c_{n-1}\not\in G_k$, $P_M(a,c_1\dots,c_{n-1})=P_M(b,c_1\dots,c_{n-1})$
}

\thm{Sufficient Condition For Equiprobability}{If $a,b\in U$ such that $K_C(a)=K_C(b)$, then $\forall c_1,\dots,c_{n-1}\not\in G_k$, $P_M(a,c_1\dots,c_{n-1})=P_M(b,c_1\dots,c_{n-1})$.}

\defn{Equiprobability (EP) Graph}{
Let $G_1,\dots,G_n$ be a set of equiprobability sets. The \textbf{Equiprobability Graph} (EP Graph) $(V_1\cup V_2,E)$ for a board $B$ is a bipartite graph defined by
\begin{align*}
    V_1&=\{G_1,\dots,G_n\}\\
    V_2&=\{a\in\mathcal{A}|C(a)=1,N(a)\in\mathbb{Z}\}\\
    E&=\{(G,a)\in V_1\times V_2\;|\;b\in G,\,G\subseteq K(a)\}
\end{align*}
where the vertices are partitioned into $V_1$, a set of EP sets, and $V_2$, the set of numbered cleared squares in $B$, such that a set and a number have an edge between them if the set lies in the neighborhood of the number (duplicate edges per the size of the $G_k$).
}

While we can approach this idea of EP groups in the same way we did perimeter enumeration before, ensuring that all nodes $a\in V_2$ have $N(a)$ edges.

TODO complete (probably won't write the algorithm since it's the same as the perimeter approach).\\


\subsection*{Is there anything better?}

Idk. Hard to believe that there's a way to compute probability exactly without enumerating the solutions, of which there are exponentially many. I could be convinced by an example though. There's probably faster approximate algorithms, but I didn't explore that route. If anyone has any optimizations for the exact algorithm, I'd be interested to know.\\

\section{Other Probabilities}

As we saw in Section \ref{sec:ssmp_inacc}, using only mine probability to guess is not perfect. However, being the first order safety probability, it's still an ok starting point for guessing. If we want to approach the optimal strategy though, higher order probabilities will need to be computed.\\

\subsection{Managing Solutions}

Last section, we got away with sweeping how solutions are managed under the rug. However, higher order probabilities will require more careful management of solutions to remain efficient. Although many ways to handle solutions result in the same runtime, through personal experience, I found that a poor implementation of solution management is often the runtime bottleneck of any probability calculation. As such, I will suggest a data structure to store solutions.\\

If we go back to the linear algebra formulation in the discussion of the exhaustive, recall that a solution can be represented as a vector of length $|\mathcal{A}|$. As such, it is fitting to store a list of solutions as a matrix where each solution is a row vector. Depending on programming language, efficient operations for matrices may be available (e.g., numpy for python).\\

If a set of solutions is treated as a row matrix, then finding all solutions with a mine at a given square is as simple as splicing the matrix on the rows where the corresponding column is one. Counting the number of mines in a solution is simply summing the columns of the matrix to a single column vector. Determining if a move can be inferred is as simple as summing the rows and checking if a resulting entry in the sum vector is equal to 0 or the number of solutions. With this, I'll leave how one would implement any other operation on a set of solutions up to the imagination of the reader.\\

\subsection{n-th Order Probability}

TODO. This is just a recursive version of the perimeter enumeration approach, but I can probably include a section on the adaptive heuristic.\\

\subsection{Approximate Algorithms}

So computing high order probabilities, is really expensive. Although single mine probability already took exponential time, $n$-th order probability takes, like, super exponential time. Since lower order probabilities are often sufficient for ``good" solvers, I believe it is still valuable to be aware of other approximate algorithms to calculate probability.\\

Since minesweeper is a stochastic decision process, it actually lends itself well to a Monte Carlo Tree Search, where probabilities can be approximated through random sampling. This can allow the higher order calculations go much deeper at the expense of accuracy. As I am relatively unfamiliar with this domain, I'll just mention that some exploration of minesweeper using MCTS has already been done\footnote{Haven't read them thoroughly, but these two papers use MCTS:\url{https://minesweepergame.com/math/consistent-belief-state-estimation-with-application-to-mines-2011.pdf} and \url{https://www.worldscientific.com/doi/abs/10.1142/S0129183120501296?journalCode=ijmpc}}. I'll leave it as an exercise to the reader to ponder on this subject more than I have.\\

\section{Logical Inference}\index{Inference!algorithms}
The previous two sections explored algorithms to compute probabilities, but playing minesweeper does not always require probabilities or guess. As we know, there are usually many instances where we know a mine exists with 0 or 1 probability without having to do any complex probability calculations. These instances were discussed deeply in Chapter \ref{sec:ng} in the form of theorems and patterns. Here, we will attempt to discuss how our knowledge of logical play can be used to construct more efficient algorithms for taking risk 0 actions.\\

For this section, the goal is to provide an implementation for a function $\texttt{Infer}$ that takes a board as input and returns two lists. One list contains squares that should be cleared ($P_M(a)=0$) and the other list contains squares that should be flagged ($P_M(a)=1$). In terms of correctness, we won't require $\texttt{Infer}$ to return every square that can be cleared or flagged with certainty, just that the sets it returns are subsets of the correct sets.\\

\subsection{Probability Approach}
Supposing we have an algorithm to compute the first order safety (i.e., mine probability), logical inference can simply be derived by finding the 0 and 1 probabilities returned by the probability algorithm.

\begin{algorithm}[h]
\caption*{Determining Logical Inference with Probability Algorithm}
\begin{algorithmic}
\Function{\texttt{Infer}}{$B=(\mathcal{A},n,M,C,N)$}
\State{$P\gets$ \texttt{Probability}$(B)$}
\State{clear, mine $\gets\emptyset$}
\For{$a$ in $\mathcal{A}$}
    \If{$P[a]$ is 0}
        \State{clear.add($a$)}
    \ElsIf{$P[a]$ is 1}
        \State{mine.add($a$)}
    \EndIf
\EndFor
\Return{clear, mine}
\EndFunction
\end{algorithmic}
\end{algorithm}

Of course, the point of this section is that we can probably do better than this approach in terms of runtime, since probability calculation takes exponential time. However, the advantage of the probability approach is that it will always find every probability 0 or 1 square, whereas the other algorithms we introduce we not always return all probability 0 or 1 squares.\\

So although, there isn't much of a point having this algorithm in logical play, we can still consider this as a valid implementation. Programmers who want a good solver, but don't want to implement any other algorithms besides \texttt{Probability}, this is the way to go.\\

\subsection{Pattern Approach}

For people who didn't read earlier sections of this document for some reason, this is the intuitive approach to implementing \texttt{Infer}. When people actually play minesweeper, they apply logic by pattern matching common scenarios.\\

Suppose we have a repository of patterns that can correspond local number configurations to a list of squares that ought to be mined and cleared. We want to know how to apply these patterns to a board.\\

Before looking how to apply each pattern to a board, let's first think about how to reduce the number of patterns we have. We can immediately reduce the number of patterns by a small factor if we notice that the rotations and mirrors of patterns can be eliminated if we rotate and mirror the board instead. TODO example\\

Another way we can reduce our number of patterns is by mine reducing the board first to remove all known mines. This allows us to only consider patterns without mines. TODO example. Since the reduction of a board to a board without mines can be useful in many other cases, the algorithm to do so is shown below.\\

\begin{algorithm}[h]
\caption*{Mine Reducing a Board}
\begin{algorithmic}
\Function{\texttt{Reduce}}{$B=(\mathcal{A},n,M,C,N)$}
\State{$N'\gets N$}
\For{$a$ in $M$}
    \For{$b$ in $K(m)$}
        \If{$N(b)\in\mathbb{Z}$}
            $N(b)\gets N(b)-1$
        \EndIf
    \EndFor
\EndFor
\Return{$B'=(\mathcal{A},n-|M|,\emptyset,C+M,N')$}
\EndFunction
\end{algorithmic}
\end{algorithm}

Now let us figure out how we can use known patterns to determine logic. I believe that an enlightening lens to look at this problem is through the lens of convolutions. Convolutions are a common mathematical operation used in signal processing were a filter is passed over all positions in a signal, and the output is roughly how well a filter matched the signal at that position. We can think of our patterns as filters, and the signal as the board.\\

So how would a convolution of a pattern over a board work? TODO (just check equality)\\

\begin{algorithm}[h]
\caption*{Determining Logical Inference with Pattern Matching using Convolution Filters}
\begin{algorithmic}
\Require{PATTERN is a list of 3-tuples (filter, mines, clears), mines and clears are relative positions with respect to filter}
\Function{\texttt{Infer}}{$B=(\mathcal{A},n,M,C,N)$}
\State{$B'\gets$ \texttt{Reduce}$(B)$}
\State{clear, mine $\gets\emptyset$}
\For{rotations and mirrors $B''$ of $B'$}
    \For{$(f,m,c)$ in PATTERNS}
        \State{$f'\gets$ conv2d$(B'',f)$}
        \For{$a$ in $\mathcal{A}$}
            \If{$f'[a]=|f|$}
                \State{mine.add($a+m$)}
                \State{clear.add($a+c$)}
            \EndIf
        \EndFor
    \EndFor
\EndFor
\Return{clear, mine}
\EndFunction
\end{algorithmic}
\end{algorithm}

TODO. Discuss channel implementation.\\

TODO. Maybe a tie in to how this approach leads into a neural network approach?\\


\subsection{MSM Theorems Approach}

TODO. Motivate

\thm{MSM Graph}{
The \textbf{MSM Graph} $(V,E)$ for a board $B$ is defined by
\begin{align*}
    V&\subset\{A\subset U\;|\;\exists k\text{ s.t. }M(A)\mapsto k\}\\
    E&=\{(A_1,A_2)\in V\times V\;|\;A_1\neq A_2, A_1\cap A_2\neq\emptyset\}
\end{align*}
where $V$ is a subset of inferable MSMs, and a pair of MSMs have an edge in $E$ if they intersect.
}

In particular, there are 4 theorems which are good enough for much of logical inference. In order of decreasing usage, they have been repeated below.

\begin{remark}
In order of decreasing usage, useful theorems for inference are
\begin{enumerate}[label=\Alph*)]
    \item Unknown Number Neighborhood MSM Theorem --- If $N(a)\in\mathbb{Z}$, then $M(K_U(a))\mapsto N(a)-|K_M(a)|$.

    \item MSM Subset Theorem --- Let $A\subset B\subset\mathcal{A}$. If $M(A)\mapsto k_A$ and $M(B)\mapsto k_B$, then $M(B\setminus A)\mapsto k_B-k_A$.

    \item MSM Disjoint-Difference Theorem --- Let $A,B_1,\dots,B_m\subset U$ such that $\forall i,j$ so $i\neq j$, $B_i\cap B_j=\emptyset$. If $M(A)\mapsto k_A$ and $M(B_i)\mapsto k_i$, then we have the following 
    \begin{enumerate}[label=\arabic*.]
        \item If $|A\setminus\bigcup_{i=1}^mB_i|=k_A-\sum_{i=1}^mk_i$\begin{itemize}
            \item $\forall a\in A\setminus\bigcup_{i=1}^mB_i$, $M(\{a\})\mapsto1$
            \item $\forall b\in(\bigcup_{i=1}^mB_i)\setminus A$, $M(\{b\})\mapsto0$
        \end{itemize}
        \item If $|\bigcup_{i=1}^mB_i\setminus A|=\sum_{i=1}^mk_i-k_A$\begin{itemize}
            \item $\forall a\in A\setminus\bigcup_{i=1}^mB_i$, $M(\{a\})\mapsto0$
            \item $\forall b\in(\bigcup_{i=1}^mB_i)\setminus A$, $M(\{b\})\mapsto1$
        \end{itemize}
    \end{enumerate}

    \item MSM Union-Subset Theorem --- Let $A,B_1,B_2\subset U$ such that $A\subset B_1\cup B_2$. If $M(A)\mapsto k_A$, $M(B_1)\mapsto k_1$, and $M(B_2)\mapsto k_2$, then we have the following \begin{enumerate}[label=\arabic*.]
        \item If $k_A=k_1+k_2$, then $\forall a\in((B_1\cup B_2)\setminus A)\cup (B_1\cap B_2)$, $M(\{a\})\mapsto 0$
        \item If $k_A=k_1+k_2-1$, then $\forall a\in(B_1\cap B_2)\setminus A$, $M(\{a\})\mapsto 0$
    \end{enumerate}
\end{enumerate}
\end{remark}

As we saw earlier in Chapter \ref{sec:ng}, A) generates all relevant first order MSMs. B) enables 1-1 patterns and dependency chains. C) result 1 enables 1-2 patterns. C) result 2 enables 2-2-2 corners. D) result 1 enables generalized 1-2-1 and 1-2-2-1 patterns. Finally, D) result 2 enables other less common patterns involving configurations of three 1s. 
While I listed 4 theorems, it is often sufficient to just implement A), B), and C) result 1 to capture most inferences.\\\\

One may notice that theorems B)-D) are all defined on general sets of MSMs. This would suggest slower than quadratic verification for the hypothesis for each of these algorithms, but as we also saw in Chapter 3, \ref{sec:ng}, these theorems are only relevant when all $B$ MSMs intersect with the $A$ MSM. This is where the MSM Graph comes in. Since first order MSMs from A) are at most of size $3\times 3$, each MSM can only intersect with MSMs with centers within 2 squares, so a total of 25 possible locations. Each location can possibly have $2^8$ different MSM (but in practice it is usually much less. These two consequences allow us to see that there are a finite number of edges (intersections per construction of the MSM graph) for each MSM node. A finite number of edges mean that each of these theorems can be verified in constant time by using the MSM graph.\\


TODO complete. Maybe rewrite pseudocode to include more details. very handwaivey atm\\

\begin{algorithm}[h]
\caption*{Determining Logical Inference with an MSM Graph}
\begin{algorithmic}
\Function{\texttt{Infer}}{$B=(\mathcal{A},n,M,C,N)$}
\State{clear, mine $\gets\emptyset$}
\State{Graph $\gets$ first order MSMs from A)}
\Repeat
    \State{Apply B), C), and D) Theorems to add nodes to Graph}
\Until{Graph did not change}
\For{Node in Graph}
    \If{Node.size == 1}
        \If{Node.k == 1}
            \State{mine.append(square in Node)}
        \EndIf
        \If{Node.k == 0}
            \State{clear.append(square in Node)}
        \EndIf
    \EndIf
\EndFor
\Return{clear, mine}
\EndFunction
\end{algorithmic}
\end{algorithm}

\subsection{Incompleteness of Logical Inference} \index{Inference!incompleteness}

So as we saw in this section, the probability approach is slow, but captures all of the squares that should be cleared and flagged. On the other hand, both the pattern matching approach and the MSM theorem approach I outlined do not necessarily find all the squares that can be cleared or flagged with certainty, but have better polynomial runtime bounds. However, as more patterns or more theorems are considered, their respective return values approach the result of the probability approach. One can then ask, is it possible to have a ``complete" set of patterns or theorems so that our polynomial-time algorithms have the same output as our exponential-time algorithm.\\

Let's consider the pattern matching approach first. It should be clear that it is indeed possible to create a complete set of patterns if the size of the board is known. This is because we can just create a complete dictionary of all the logic deducible steps as patterns. However, it should also be clear that the number of patterns if we do this will grow exponentially as the size of the board increases. As such, the pattern approach can never approach a complete solution while maintaining polynomial time.\\

Now let's consider the MSM theorem approach. Like the pattern approach, we can easily have an exponential number of theorems to have complete logical play, by simply having a theorem for each pattern. We can also easily easily define  theorems that takes exponential time to compute to complete logical play (For instance, $P_M(a)=0\implies P_M(\{a\})\mapsto 0$ takes exponential time to compute, but it along with its counterpart form a finite set of theorems that complete logical play). As such, we want to know if there is a polynomial number of theorems that run in polynomial time that can complete logical inference.\\

As it turns out, nobody knows. A well-known result by Richard Kaye\footnote{\url{https://link.springer.com/article/10.1007/BF03025367}} made back in 1998 was that the minesweeper consistency problem is NP-complete. The consistency problem is simply determining if a given board $B$ is consistent (i.e., continues to a complete board).\\

While this is interesting, it is not immediately useful to us the player, who already knows that the board we are playing is consistent. A more pertinant result from Allan Scott\footnote{\url{https://link.springer.com/article/10.1007/s00283-011-9256-x}} in 2011 is that the minesweeper inference problem is co-NP complete. In math, the minesweeper inference problem is similarly defined where given a board $B$, the task is to provide a square $a\in U$ such that $P_M(a)$ is 0 or 1, or state that no square exists. Evidently, this is a just a version of the function we want to find, \texttt{Infer}. One can see how these two problem formulations reduce to each other easily.\\

Both Kaye and Scott show that the consistency and inference problems are hard by reducing satisfiability (and unsatisfiability) to solving minesweeper through the construction of Boolean circuits in minesweeper. Those interested should check those papers out, but here, we only care about the result. Since the minesweeper inference problem is co-NP complete and we don't yet know if P=NP=co-NP (or even if NP=co-NP), as far as we know there is no polynomial time algorithm that can get the exact same answer as the probability approach to the problem. Although, if you, the reader, do find a polynomial algorithm for \texttt{Infer} with respect to board size, hit me up :)\\

\section{Solvers}

TODO. Just use the building blocks mentioned up until now.\\