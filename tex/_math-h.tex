\usepackage{amsmath}
\usepackage{amsthm}
\usepackage{bbm}

\makeatletter
\def\thm@space@setup{%
  \thm@preskip=\parskip \thm@postskip=0pt
}

% Math commands
\newenvironment{sysmatrix}[1]
 {\left[\begin{array}{@{}#1@{}}}
 {\end{array}\right]}
\newcommand{\ro}[1]{%
  \xrightarrow{\mathmakebox[\rowidth]{#1}}%
}
\newlength{\rowidth}% row operation width
\newcommand{\vt}[3]{\begin{sysmatrix}{c}#1\\#2\\#3\end{sysmatrix}}
\AtBeginDocument{\setlength{\rowidth}{3em}}
\newcommand\bigzero{\makebox(0,0){\text{\huge0}}}

\newcommand*\Eval[3]{\left.#1\right\rvert_{#2}^{#3}}
\newcommand\norm[1]{\left\lVert#1\right\rVert}

\newcommand{\cov}[1]{\text{Cov}\left(#1\right)}
\newcommand{\ev}[1]{E\left[#1\right]}

\newcommand{\argmin}{\mathop{\mathrm{arg\,min}}\limits}
\newcommand{\argmax}{\mathop{\mathrm{arg\,max}}\limits}


% Theorems + other boxed stuff (adapted from gilles castel)
\makeatother
\usepackage{thmtools}
\usepackage[framemethod=TikZ]{mdframed}
\mdfsetup{skipabove=1em,skipbelow=0em}

\declaretheoremstyle[
    headfont=\bfseries\sffamily\color{ForestGreen!70!black}, 
    notefont=\bfseries\sffamily\color{ForestGreen!70!black}, 
    bodyfont=\normalfont,
    notebraces={( }{ )},
    headpunct={},
    postheadspace=1em,
    mdframed={
        linewidth=2pt,
        rightline=false, topline=false, bottomline=false,
        linecolor=ForestGreen, backgroundcolor=ForestGreen!5,
    }
]{defshaded}

\declaretheoremstyle[
    headfont=\bfseries\sffamily\color{NavyBlue!70!black}, 
    notefont=\bfseries\sffamily\color{NavyBlue!70!black}, 
    bodyfont=\normalfont,
    notebraces={( }{ )},
    headpunct={},
    postheadspace=1em,
    mdframed={
        linewidth=2pt,
        linecolor=NavyBlue
    }
]{egbox}

\declaretheoremstyle[
    headfont=\bfseries\color{RawSienna!70!black}, 
    notefont=\bfseries\color{RawSienna!70!black}, 
    bodyfont=\normalfont,
    notebraces={( }{ )},
    headpunct={},
    postheadspace=1em,
    mdframed={
        linewidth=2pt,
        rightline=false, topline=false, bottomline=false,
        linecolor=RawSienna, backgroundcolor=RawSienna!5,
    }
]{thmshaded}

\declaretheoremstyle[
    headfont=\bfseries\sffamily\color{RawSienna!70!black}, 
    notefont=\bfseries\sffamily\color{RawSienna!70!black}, 
    bodyfont=\normalfont,
    notebraces={( }{ )},
    headpunct={},
    postheadspace=1em,
    mdframed={
        linewidth=2pt,
        rightline=false, topline=false, bottomline=false,
        linecolor=RawSienna
    }
]{thmline}

\declaretheoremstyle[
    headfont=\bfseries\sffamily\color{RawSienna!70!black}, 
    notefont=\bfseries\sffamily\color{RawSienna!70!black}, 
    bodyfont=\normalfont,
    notebraces={( }{ )},
    headpunct={},
    postheadspace=1em,
    mdframed={
        linewidth=2pt,
        rightline=false, topline=false, bottomline=false,
        linecolor=RawSienna, backgroundcolor=RawSienna!1,
    },
    qed=\qedsymbol
]{thmproof}

\theoremstyle{definition}

\declaretheorem[style=defshaded, numbered=no, name=Definition]{definition}
\declaretheorem[style=egbox, numberwithin=chapter, name=Example]{example}
\declaretheorem[style=thmshaded, numberwithin=chapter, name=Theorem]{theorem}
\declaretheorem[style=thmshaded, numberwithin=chapter, sibling=theorem, name=Proposition]{proposition}
\declaretheorem[style=thmshaded, numberwithin=chapter, sibling=theorem, name=Lemma]{lemma}
\declaretheorem[style=thmshaded, numberwithin=chapter, sibling=theorem, name=Corollary]{corollary}
\declaretheorem[style=thmline, numbered=no, name=Remark]{remark}
\declaretheorem[style=thmline, numbered=no, name=Note]{note}

\declaretheorem[style=thmproof, numbered=no, name=Proof]{replacementproof}
\renewenvironment{proof}[1][\proofname]{\vspace{-1.1em}\begin{replacementproof}}{\end{replacementproof}}
\AtBeginEnvironment{proof}{\renewcommand{\qedsymbol}{}}{}{}

\newcommand{\eg}[2]{
    \begin{example}[#1]\phantom{.}\newline#2\end{example}
}
\newcommand{\thm}[2]{
    \begin{theorem}[#1]\phantom{.}\newline#2\end{theorem}
}
\newcommand{\cor}[2]{
    \begin{corollary}[#1]\phantom{.}\newline#2\end{corollary}
}
\newcommand{\lem}[2]{
    \begin{lemma}[#1]\phantom{.}\newline#2\end{lemma}
}
\newcommand{\defn}[2]{
    \begin{definition}[#1]\phantom{.}\newline#2\end{definition}
}


% misc
\newcommand{\shrug}[1][]{%
\begin{tikzpicture}[baseline,x=0.8\ht\strutbox,y=0.8\ht\strutbox,line width=0.125ex,#1]
\def\arm{(-2.5,0.95) to (-2,0.95) (-1.9,1) to (-1.5,0) (-1.35,0) to (-0.8,0)};
\draw \arm;
\draw[xscale=-1] \arm;
\def\headpart{(0.6,0) arc[start angle=-40, end angle=40,x radius=0.6,y radius=0.8]};
\draw \headpart;
\draw[xscale=-1] \headpart;
\def\eye{(-0.075,0.15) .. controls (0.02,0) .. (0.075,-0.15)};
\draw[shift={(-0.3,0.8)}] \eye;
\draw[shift={(0,0.85)}] \eye;
\draw (-0.1,0.2) to [out=15,in=-100] (0.4,0.95); 
\end{tikzpicture}}